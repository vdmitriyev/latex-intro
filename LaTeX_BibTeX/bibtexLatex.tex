\documentclass[slidestop,compress,mathserif]{beamer}
\usetheme{Frankfurt}
\usecolortheme{seagull}

% Load packages
\usepackage{alltt}
\usepackage{verbatim}
\usepackage{geometry}                % See geometry.pdf to learn the layout options. There are lots.
\usepackage{graphicx}
\usepackage{amssymb}
\usepackage{amsmath}
\usepackage{epstopdf}
\usepackage{verbatim}
%\usepackage{musixtex}
%\usepackage{xymtexps}
\usepackage{feynmp}

%\usepackage{pgfpages}
%\pgfpagesuselayout{4 on 1}[letterpaper,landscape,border shrink=5mm]

\DeclareGraphicsRule{.tif}{png}{.png}{`convert #1 `dirname #1`/`basename #1 .tif`.png}


% Title 
% Note: [short title]{long title}, [short author(s) name]{long author(s) name}
\title{BibTeX and LaTeX}
\subtitle{}
\author{Viktor Dmitriyev} 
\institute{Adapter from Mini Course on LaTeX by \href{https://github.com/OpenIntroOrg/mini-course-materials}{David Diez}}
\date{}

\begin{document}
\definecolor{highlight}{rgb}{.7,.1,.1}
\definecolor{command}{rgb}{.1,.1,.9}
\definecolor{comment}{rgb}{1,0,0}
\definecolor{braces}{rgb}{0,0.5,0}
\newenvironment{act}[1]{{\color{command}#1}}{}
\newcommand{\lcom}[1]{{\color{command}$\backslash$#1}}
\newcommand{\larg}[1]{{\color{braces}$\{${\color{black}#1}$\}$}}
\newcommand{\mathText}[1]{{\color{braces}\${\color{black}#1}\$}}


\frame{ \titlepage }

\begin{frame}
  \frametitle{Outline}
  \begin{itemize}
  \item BibTeX: bibliographies in LaTeX
  \end{itemize}
\end{frame}

\begin{frame}  \frametitle{Guide to LaTeX}
	The book \textit{Guide to LaTeX} offers a very nice introduction, and we will closely follow some of the examples in these chapters in this class:
	\begin{itemize}
	\item[11,12] BibTeX
	\end{itemize}
%If you are looking for a LaTeX resource, \textit{Guide to LaTeX} is a good choice.
\end{frame}

\part{}

\include{bibtex/bibtex}

\section[Wrap-up]{Wrap-up}
\subsection[Wrap-up]{Wrap-up}

\begin{frame}  \frametitle{Wrap-up}
After this class, you should have a general idea of
\vspace{1mm} \\
\begin{itemize}
	\item creating bibliographies using BibTeX
\end{itemize}
\vspace{1mm}
Any questions?
\end{frame}





\end{document}